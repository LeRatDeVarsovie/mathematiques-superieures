\documentclass{article}
\usepackage[utf8]{inputenc}
\usepackage{geometry}
\usepackage{tcolorbox}
\usepackage{amsmath, amssymb}
\renewcommand{\labelitemi}{$\ast$}
\usepackage{stmaryrd}
\usepackage{parskip}
\usepackage{hyperref}

\geometry{
 a4paper,
 total={170mm,257mm},
 left=20mm,
 top=20mm,
 }

 \hypersetup{
    colorlinks=true,
    linkcolor=black
}

\renewcommand{\thesubsection}{\Roman{subsection}}

\usepackage{graphicx}
\usepackage{titling}
\title{\textbf{Théorèmes de Sup :}}
\author{Loù et Henri}
\date{2025-2026}
\usepackage{fancyhdr}
\fancypagestyle{plain}{% the preset of fancyhdr
\fancyhf{} % clear all header and footer fields
\fancyfoot[L]{\thedate}
\fancyhead[L]{Théorèmes de Sup}
\fancyhead[R]{\theauthor}
}
\makeatletter
\def\@maketitle{%
\newpage
\null
\vskip 1em%
\begin{center}%
\let \footnote \thanks
 {\LARGE \@title \par}%
\vskip 1em%
%{\large \@date}%
\end{center}%
\par
\vskip 1em}
\makeatother
\usepackage{lipsum}

\begin{document}

\maketitle

\subsection{. \textbf{Quelques résultats d'algèbre}}

\hyperlink{page.2}{Théorème de Lagrange \dotfill 2}\\
\hyperlink{page.3}{Sous-groupes additifs de $\mathbb{Z}$ \dotfill 3}\\
\hyperlink{page.4}{Groupe fini de cardinal premier \dotfill 4}\\
\hyperlink{page.4}{Une pseudo-réciproque du résultat précédent \dotfill 4}\\


\newpage

\begin{tcolorbox}[colback=gray!10, colframe=black, boxrule=1.5pt, arc=0pt,
                  left=4pt, right=4pt, top=4pt, bottom=4pt,
                  leftrule=1.5pt, rightrule=0pt, toprule=0pt, bottomrule=0pt]
 \textbf{\underline{Théorème de Lagrange} :} \\
\\
 Soit $G$ un groupe fini et $H$ un sous-groupe de $G$, alors l'ordre de $H$ divise l'ordre de $G$.
\end{tcolorbox}

\textit{Preuve :} Considérons la relation binaire $R$ de $G$ définie par :
$$\forall (x,y) \in G^2, \thinspace xRy \iff y \in \{x \cdot h \mid h \in H \}$$
Montrons que $R$ est une relation d'équivalence sur $G$ :\\
\begin{itemize}

\item \textbf{\underline{Réflexivité} :} Soit $x \in G$, $xRx \iff x \in \{x \cdot h \mid h \in H\}$ qui est une assertion toujours vérifiée, il suffit de prendre $h = e$ 
(où $e$ est l'élément neutre de $G$).
\\
\item \textbf{\underline{Symétrie} :} Si pour $(x,y) \in G^2$, on a $xRy \iff y \in \{x \cdot h \mid h \in H \}$ alors pour tout 
$y \in G$ il existe $h \in H$ tel que $y = x \cdot h$ donc $x = y \cdot \underbrace{h^{-1}}_{\in H}$ ($h^{-1}$ existe car $G$ est un groupe) ainsi $x \in \{y \cdot h' \mid h' \in H \}$
donc on a bien $yRx$ d'où la symétrie de $R$.
\\
\item \textbf{\underline{Transitivité} :} Supposons que pour $x,y,z \in G$ on ai $xRy$ et $yRz$ alors $y \in \{x \cdot h \mid h \in H \}$ et \\
$z \in \{y \cdot h' \mid h' \in H \}$. Alors $z \in \{x \cdot \underbrace{(h \cdot h')}_{\in H} \mid h,h' \in H \}$ c'est à dire :
$$z \in \{x \cdot h'' \mid h'' \in H \}$$
\end{itemize}
Donc $xRz$ d'où la transitivité de $R$. On a donc bien montré que $R$ était une relation d'équivalence sur $G$.\\

Soit $a \in G$, on cherche à déterminer :
$$\overline{a} := \{b \in G \mid bRa\}$$
On a $x \in \overline{a} \iff xRa \iff a \in \{x \cdot h \mid h \in H\} \iff \exists h \in H, \thinspace a = x \cdot h \iff \exists h \in H, \thinspace x = a \cdot h^{-1}$
On a donc montré par double-inclusion que $\overline{a} = a \cdot H$.\\
\\
Il est connu que la famille $(\overline{x}_i)_{i \in \llbracket 1, p \rrbracket}$ 
(où les $\overline{x}_i$ sont des classes d'équivalences pour la relation $R$ deux à deux distinctes) forme une partition de $G$ ($1 \leq p \leq \text{Card}(G)$).
On a donc :
$$\bigsqcup_{i = 1}^p \overline{x}_i = G$$
Il vient alors que :
$$\text{Card}(\bigsqcup_{i = 1}^p \overline{x}_i) = \text{Card}(G)$$
Donc :
$$\sum_{i = 1}^p \underbrace{\text{Card}(\overline{x}_i)}_{\text{Card}(H)} = \text{Card}(G)$$
(Le fait que $\text{Card}(\overline{x}_i) = \text{Card}(H)$ pour tout $i \in \llbracket 1, p \rrbracket$ est assez intuitif et se justifie très bien en exhibant une bijection de $\overline{x}_i$ dans $H$ !)\\
\\
Enfin :
$$p \times \text{Card}(H) = \text{Card}(G)$$
D'où le théorème de Lagrange. $\thinspace \square$

\newpage

\begin{tcolorbox}[colback=gray!10, colframe=black, boxrule=1.5pt, arc=0pt,
                  left=4pt, right=4pt, top=4pt, bottom=4pt,
                  leftrule=1.5pt, rightrule=0pt, toprule=0pt, bottomrule=0pt]
 \textbf{\underline{Sous-groupes additifs de $\mathbb{Z}$} :} \\
\\
 blabla skibidi brainrot i'm edging
\end{tcolorbox}

\textit{Preuve :}




\newpage

\begin{tcolorbox}[colback=gray!10, colframe=black, boxrule=1.5pt, arc=0pt,
                  left=4pt, right=4pt, top=4pt, bottom=4pt,
                  leftrule=1.5pt, rightrule=0pt, toprule=0pt, bottomrule=0pt]
 \textbf{\underline{Théorème} :} \\
\\
 Soit $(G, \cdot)$ un groupe fini d'ordre $p$ premier, alors $G$ est cyclique.
\end{tcolorbox}

\textit{Preuve :}

Soit $(G,\cdot)$ un groupe fini de cardinal $p$ premier, on note $e$ sont élément neutre.\\
On considère $x \in G$ différent de $e$ (ce qui est à priori licite car $\text{Card}(G) \geq 2$).\\
Dans ce cas le sous-groupe engendré par $x$ (noté $\langle x \rangle$ dans tout ce document) est égal à $G$ car $\text{Card}(\langle x \rangle) \mid p$ (par le théorème de Lagrange). En effet, comme $p$ est premier, ceci ne laisse que deux possibilités :\\
soit $\langle x \rangle = \{e\}$ ce qui est impossible, soit $\langle x \rangle = G$.\\
On obtient donc que $x$ est un générateur de $G$, ainsi $G$ est monogène et fini donc cyclique. $\thinspace \square$

\vspace{1cm}

\begin{tcolorbox}[colback=gray!10, colframe=black, boxrule=1.5pt, arc=0pt,
                  left=4pt, right=4pt, top=4pt, bottom=4pt,
                  leftrule=1.5pt, rightrule=0pt, toprule=0pt, bottomrule=0pt]
 \textbf{\underline{Une pseudo-réciproque du résultat précédent} :} \\
\\
 Soit $(G, \cdot)$ un groupe de neutre $e$ dont les seuls sous-groupes sont $\{e\}$ et $G$, alors $G$ est cyclique d'ordre un nombre premier.
\end{tcolorbox}

\end{document}